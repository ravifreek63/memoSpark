\section{Related Work}
\label{sec:related}
\paragraph{}
We build our design on top of the Spark \cite{zaharia2012resilient}
runtime engine. While Spark provides useful abstractions for performing
distributed computations on top of RDDs, it lacks good support for
performing queries (such as \texttt{filter}, \texttt{count}, etc.) on
pre-specified keys.  Other frameworks such as Shark
\cite{engle2012shark}, Blink-DB \cite{agarwal2013blinkdb} have been used
for performing queries on top of Spark. However, they require data to be
completely structured as like data stored in RDBMS. We have designed our
system for semi-structured data, which consists of key-value pairs in
data files. 

\paragraph{}
Systems such as MANIMAL \cite{jahani2011automatic} perform automatic
optimizations on top of MapReduce programs. They can provide similar
optimizations by building B+ trees on data already stored within the
HDFS. However, our design works for an in-memory runtime engine, though
fundamentally it has some similar ideas.

\paragraph{}
Systems such as SSDAlloc\cite{badam2011ssdalloc} have looked at
detecting object-level accesses within native language programs. The
system relies on page protection mechanism which can be very expensive.
Our system intercepts object level accesses within a managed runtime and
is completely transparent, therefore addresses a different problem
subspace.
