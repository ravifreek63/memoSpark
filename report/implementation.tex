\section{Implementation} 
\label{sec:implementation}
\paragraph{}
We here describe the implementation of our indexed RDDs. We create a
separate derived class \texttt{IndexedRDDKV}, with the RDD class as its
base class. We support the following interfaces on the RDD class:

\begin{enumerate}
\item \texttt{indexedKV()}: The driver program can call this method on
    any RDD to transform the given RDD into an RDD of indexed RDD with
    support of indexing. Each record must be a key-value pair. 
    We expose the following interfaces on the \texttt{IndexedRDDKV}
    class:
\item \texttt{rangePartitions()}: The method creates a range partitioned
    index on the data in the RDDs. For each partition, the smallest and
    the largest values are stored as the range of that partition.
\item \texttt{searchByKeyRangePartitioned(key:String)}: This method
    searches for the key within the range partitions. The master first
    filters the set of partitions within which the key can fall into and
    then runs the query on those select partitions. 
\item \texttt{indexPartitions()}: This method is used to
    create a partition on the range of keys when the input data is
    already sorted. 
\item \texttt{searchByKey(key:String)}: The method searches the given
    key within the RDD using the index created by the
    \texttt{indexedPartitions} function. 
\end{enumerate}
