% -*-LaTeX-*-
% $Id: abstract.tex 70 2007-01-30 21:59:16Z nicolosi $

\begin{abstract}
Whislt the advent of in-memory computation platforms for big-data
applications has considerably improved application throughput, it has
also significantly increased the reliance on the available memory
resources within individual computing nodes.  The current design of
large scale data computation platforms makes use of commodity servers,
which typically have limited resources on each node.  In this paper we
present an analysis of the impact of this tension that could possibly
hamper future big-data computing platforms.  To reveal the effects of
memory pressure on big-data applications running on Spark, we conduct
experiments to quantify the impact of garbage collection done by storage
management on the application throughput. Besides, we present results
based on object level access patterns of an application running on Spark
through a unique interception mechanism oblivious to applications.
Additionally, we extend the abstraction of RDDs by providing indexing
capabilities within key-valued RDDs. We show that simple range
partitioning of real world log data can help reduce query time when
using RDDs. 
\end{abstract}

