\section{Motivation} 
\label{sec:motivation}
\paragraph{}
With the advent of Spark like Big data runtimes big data, workloads will create excessive pressure on the current memory subsystems. We believe that current memory subsystems are not well suited to handle spikes of memory requirement and will get severely affected by non-uniform memory requirement patterns. Fig \ref{fig:fig1} shows the overall time it takes for the Garbage Collector for different tasks when running a page ranking application. The initial tasks require a large amount of memory. We believe this could be because of the extra memory required when reading data from the secondary storage (due to deserialization of data). There is a subsequent drop in the throughput for these specific set of tasks. This was the primary motivation for studying the effect of memory pressure on different configurations. Besides, we believe, the answer to better memory management lies in understanding the underlying access patterns of different workloads and offloading low priority data on hybrid memory subsystems. Therefore, we performed experiments quantifying access patterns in baseline spark system.

\begin{figure}[!ht]
\caption{Overall time spent in garbage collection for tasks with 2GB and 6GB of memory per node}
\label{fig:fig1}
\includegraphics[scale=0.50]{./images/image1.png}
\end{figure}

\paragraph{}
	Another, observation we had is that the abstraction of RDDs is still inept when handling search queries on semi-structured datasets such as logs from applications, page views etc. When filtering a dataset Spark has to parse all the partitions within a dataset, while the actual results might be contained in a subset of the overall partitions. Fig \ref{fig:fig2} shows the comparison between Vanilla Spark's filtering query and an ideal implementation of a set of queries on a dump from Wikimedia \cite{wikimedia} where we observe that on an average only 48\% of the total partitions contained relevant results. This was the primary motivation behind building a range partitioning mechanism on top of partitions and thereby indexing the data stored in RDDs.

\begin{figure}[!ht]
\caption{Comparison of query runtimes on Vanilla (V-Spark) and an ideal query engine (I-Spark)}
\label{fig:fig2}
\includegraphics[scale=0.50]{./images/image2.png}
\end{figure}

